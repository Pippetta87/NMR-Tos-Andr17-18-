\documentclass[10pt,xcolor={usenames},fleqn,mathserif,serif]{beamer}
%%%Usefull link
%tikz-equations:
%http://www.wekaleamstudios.co.uk/posts/creating-a-presentation-with-latex-beamer-equations-and-tikz/
\hypersetup{pdfpagemode=FullScreen}
%% colors
\definecolor{bittersweet}{rgb}{1.0, 0.44, 0.37}
\definecolor{brilliantlavender}{rgb}{0.96, 0.73, 1.0}
\definecolor{antiquefuchsia}{rgb}{0.57, 0.36, 0.51}
\definecolor{violetw}{rgb}{0.93, 0.51, 0.93}
\definecolor{Veronica}{rgb}{0.63, 0.36, 0.94}
\definecolor{atomictangerine}{rgb}{1.0, 0.6, 0.4}
\definecolor{darkgray}{rgb}{0.66, 0.66, 0.66}
\definecolor{brightcerulean}{rgb}{0.11, 0.67, 0.84}
\definecolor{cadmiumorange}{rgb}{0.93, 0.53, 0.18}
\definecolor{ochre}{rgb}{0.8, 0.47, 0.13}
\definecolor{midnightblue}{rgb}{0.1, 0.1, 0.44}
\definecolor{lemon}{rgb}{1.0, 0.97, 0.0}
\definecolor{grey}{rgb}{0.7, 0.75, 0.71}
\definecolor{amber}{rgb}{1.0, 0.75, 0.0}
\definecolor{almond}{rgb}{0.94, 0.87, 0.8}
\definecolor{bf}{RGB}{88, 86, 88}
\definecolor{bb}{RGB}{177, 177, 177}
%%%%%%%%%%%%%%%%%%%%%%%%%%%%%%%%%%% importa pacchetti
\usepackage{usepkg}
%%%%%%%%%%%%%%%%%%%%%%%%%%%%%%%%%%% Funzioni generali
\usepackage{functions}
%http://tex.stackexchange.com/questions/246/when-should-i-use-input-vs-include
\newcommand{\setmuskip}[2]{#1=#2\relax} %%problem usinig mu with calc (req by mathtools) loaded
\usepackage{sources}
%\usepackage{length}
%%%%%%%%%%%%%%%%%%%%%%%%%%%%%%%%%%% Funzioni per questo file main
\usepackage{mathOp}
\usepackage{beamersetup}
\def\status{coazione}
\def\keeptrying{coazione}
\usepackage{LocalF}
%%%%%%%%%%%%%%%%%%%%%%%%%%%%%%%%%
\title{Risonanza magnetica nucleare}
% A subtitle is optional and this may be deleted
\subtitle{Risonanza magnetica nucleare}
\date{NOW, \today}
% - Either use conference name or its abbreviation.
% - Not really informative to the audience, more for people (including
%   yourself) who are reading the slides online
% Let's get started
\begin{document}

\addtobeamertemplate{block begin}{\setlength\abovedisplayskip{2pt}\setlength\belowdisplayskip{2pt}\setlength\abovedisplayshortskip{2pt}\setlength\belowdisplayshortskip{2pt}}

\addtobeamertemplate{block begin}{\vspace*{-3pt}}{}
\addtobeamertemplate{block end}{}{\vspace*{-3pt}}

\begin{frame}
  \titlepage
\end{frame}

\begin{frame}{tempo interno per}
\tableofcontents[onlyparts]
\end{frame}

% Section and subsections will appear in the presentation overview
% and table of contents.
%\frame{\tableofcontents[onlyparts]}
%\begin{frame}{Argomenti}
%  \tableofcontents[part=1,hideallsubsections%,pausesections
%  ]
%  % You might wish to add the option [pausesections]
%\end{frame}

\part{Intro}\linkdest{intro}

\begin{frame}[label={why}]{Why?}
Cardiologo: adesso non riesco a trovare essere me - il mio nemico \'e il sonno - workout condizione stanchezza (l'autunno/inverno scorso passato a pensare di fare e essere coazione).
Uso del tempo sempre come 2 giorni prima dell'esame.
Come essere me? Capitalit\'a progetto
\end{frame}

\begin{wordonframe}{Cosa provo?}
Angoscia insopprimibile: essenza me nella dipendenza
Abitudine me ''pensare''
Cosa provo mentre ascolto Tosetti che legge la presentazione?
Cosa provo nel malessere dietro il magnete?
Cosa provo quando devo iniziare a fare qualcosa?
Quando non faccio in tempo a fare le cose che dovrei: 
\end{wordonframe}

\part{NMR blowing}
\begin{frame}[allowframebreaks]{Reg Lez}

\begin{itemize}
  

\item 

\end{itemize}

\end{frame}

\part{Succo}\linkdest{succo}
\begin{frame}{Induced fem in coil due to spinning M}
\begin{align*}
&emf=-\TDof{t}\int d^3r\vec{M}(\vec{r},t)\cdot\vec{\mathcal{B}}_{rec}\\
&S\propto \frac{\gamma^3B_0^2\rho_0}{T}
\end{align*}
\begin{figure}[!ht]\includegraphics[trim={0cm 0cm 0 0},clip, keepaspectratio,width=0.5\textwidth]{nuclei}\label{fig:nuclei}\end{figure}
\end{frame}

\begin{frame}{Equazione di Bloch: sistema Laboratorio/rotante}
Sistema laboratorio:
\begin{equation*}
\TDy{t}{\vec{M}(t)}=\gamma\vec{M}\wedge(\vec{B}_0+\vec{B}_1)-\frac{M_x\hat{x}}{T_2}-\frac{M_y\hat{y}}{T_2}-\frac{(M_z-M_0)\hat{z}}{T_1}
\end{equation*}
Sistema rotante con $\Omega=\omega$, $\Delta\omega=\omega-\omega_0$:
\begin{equation*}
\PDy{t}{\vec{M}(t)}=\gamma\vec{M}\wedge(\Delta\omega\hat{z}-\omega_1\hat{x})-\frac{M_x\hat{x}}{T_2}-\frac{M_y\hat{y}}{T_2}-\frac{(M_z-M_0)\hat{z}}{T_1}
\end{equation*}
-Omogeneit\'a sistema: solo $T_1$ e $T_2$. -Dipoli poco interagenti (D. Boltzman). -Netta separazione scala microscopica/macroscopica: molti atti elementari costituiscono $T_1$ e $T_2$.
\begin{columns}[T]
\begin{column}{0.5\textwidth}
\begin{block}{On resonance}
    Per $\omega=\omega_0$:
\begin{align*}
&(\TDy{t}{\vec{\mu}})'=\omega_1\vec{\mu}\wedge\hat{x}'
\end{align*}
Flip-angle: $\Delta\theta=\gamma B_1\tau$
\end{block}
\end{column}
\begin{column}{0.5\textwidth}
\begin{align*}
&\PDy{t}{M_x}=\PDy{t}{u}=\frac{-u}{T_2}+v\Delta\omega\\
&\PDy{t}{M_y}=\PDy{t}{v}=\frac{-v}{T_2}-u\Delta\omega-\omega_1M_z\\
&\PDy{t}{M_z}=\omega_1 v-\frac{(M_z-M_0)}{T_1}
\end{align*}
\end{column}
\end{columns}
\end{frame}

\begin{frame}{Soluzioni equazioni di Bloch stazionarie}
\begin{columns}[T]
\begin{column}{0.5\textwidth}
\begin{block}{Equazioni stazionarie}
\begin{align*}
&u-vT_2=0\\
&v+uT_2\Delta\omega+\omega_1T_2M_z=0\\
&M_0=-\omega_1T_1v+M_z
\end{align*}
\end{block}
\end{column}
\begin{column}{0.5\textwidth}
\begin{align*}
    &u=M_0\frac{\omega_1\Delta\omega T_2^2}{1+(T_2\Delta\omega)^2+\omega_1^2T_1T_2}\\
    &u=M_0\frac{\omega_1T_2}{1+(T_2\Delta\omega)^2+\omega_1^2T_1T_2}\\
    &M_z=M_0\frac{1+(T_2\Delta\omega)^2}{1+(T_2\Delta\omega)^2+\omega_1^2T_1T_2}\\
    \end{align*}
\end{column}
\end{columns}
    Nel riferimento del Lab la magnetizzazione ruota con velocit\'a angolare costante:
    \begin{align*}
    &M_x=u\cos{\omega t}-v\sin{\omega t}=\sqrt{u^2+v^2}\cos{(\omega t-\phi)}\\
    &M_y=u\cos{\omega t}+v\sin{\omega t}=\sqrt{u^2+v^2}\sin{(\omega t-\phi)}\\
    &M_z=M_z
\end{align*}
\end{frame}

\begin{frame}{Situazione intermedia: $\omega_1T_{1,2}\approx1$, $\Delta\omega\approx0$ intorno.}
\begin{columns}[T]
\begin{column}{0.5\textwidth}
\begin{figure}
    \centering
    \includegraphics[width=0.99\textwidth,keepaspectratio]{blochstationary}
    \label{fig:blochstationary}
\end{figure}
\end{column}
\begin{column}{0.5\textwidth}
Variazioni in funzione di $\Delta\omega$ nel sistema rotante. $T_1=T_2=T_R$ - regime di semi-saturazione.
Magnetizzazione complessa per descrivere dissipazione: $M_{x/y}=(\chi'-j\chi'')H_{1,x/y}$. Infatti:
\begin{align*}
&M_x=u\cos{(\omega t)}-v\sin{(\omega t)}\\
&=u\frac{B_{1x}}{B_1}-v\frac{B_{1y}}{B_1}=u\frac{B_{1x}}{B_1}-vj\frac{B_{1x}}{B_1}\\
&=\mu_0(a-jb)H_{1,x}
\end{align*}
\end{column}
\end{columns}
\end{frame}

\begin{frame}{Equazione di Block: impulsive excitation}
$T_2*=T_2+T_2$: se domina $T_2'$ dovuto a disomogeneit\'a compo magnetico esterno $\magort{}$ pu\'o essere rifasata.
Interazioni spin-lattice: $T_1$.
\begin{align*}
&\TDy{t}{M_z}=\frac{(M_0-M_z)}{T_1}\\
&M_z(t)=M_z(0)\exp{-\frac{t}{T_1}}+M_0(1-\exp{-\frac{t}{T_1}})
\end{align*}
Local field variation: dephasing $T_2$.
\begin{align*}
&\TDy{t}{\magort}=(\gamma\magort{}\wedge\vec{B}_{ext})_{NR}-\frac{1}{T_2}\magort{}\\
&\magort{}(t)=\magort{}(0)\exp{-\frac{t}{T_2}}
\end{align*}
\end{frame}

\begin{frame}{Relaxation for spin $1/2$: $T_1$}
\begin{columns}[T]
\begin{column}{0.5\textwidth}
$\vec{H^*}$: random magnetic field
\begin{align*}
&\TDy{t}{\vec{S}}=\gamma\vec{S}\wedge\vec{H}_0+\gamma\vec{S}\wedge\vec{H^*}
\end{align*}
Le componenti di $\vec{H^*}$ non sono correlate, $\exv{H_x^2}=\exv{H_y^2}$, la funzione di autocorrelazione di una componente \'e $\exv{H_x^*(t+\tau)H_x^*(t)}=\exv{H_x^{*2}}\exp{-\tau/\tau_c}$.
L'equazione per $S_z$ riferita al rif. Lab, e con $t_1=t$, $t_2=\tau$:
\end{column}
\begin{column}{0.5\textwidth}
Soluzioni partendo da $S_z(0)\neq0$, $S_x(0)=S_y(0)=0$:
\begin{align*}
&S_y(t_1)=-\gamma S_z(0)\int_0^{t_1}H_y^*(t_2)d\,t_2\\
&S_y(t_1)=\gamma S_z(0)\int_0^{t_1}H_x^*(t_2)d\,t_2\\
&S_z(t_1)=S_z(0)\\
&S_z(T)=S_z(0)+\\
&+\gamma\int_0^Td\,t_1[S_x(t_1)H_y^*(t_1)-S_y(t_1)H_x^*(t_1)]
\end{align*}
\end{column}
\end{columns}
\begin{align*}
&S_z(T)-S_z(0)=-\gamma^2S_z(0)(\exv{H_x^{*2}}+\exv{H_y^{*2}})\int_0^Td\,t\int_0^td\,\tau\cos{(\omega_0\tau)}\exp{-\tau/\tau_c}\\
&\to[S_z(T)-S_z(0)]=-\frac{TS_z(0)}{T_1}\\
&\frac{1}{T_1}=\gamma^2(\exv{H_x^{*2}}+\exv{H_y^{*2}})\frac{\tau_c}{1+\omega_0^2\tau_c^2}
\end{align*}
\end{frame}

\begin{frame}{Relaxation for spin $1/2$: $T_2$}
\begin{align*}
&S_x(T)-S_x(0)=-\gamma^2S_x(0)\int_0^Td\,t_1\int_0^{t_1}d\,t_2[H_y^*(t_1)H_y^*(t_2)+H_z^*(t_1)H_z^*(t_2)]\\
&\frac{1}{T_2}=\gamma^2[\tau_c\exv{H_z^{*2}}+\frac{1}{2}(\exv{H_x^{*2}}+\exv{H_y^{*2}})\frac{\tau_c}{1+\omega_0^2\tau_c^2}]
\end{align*}
\end{frame}

\begin{wordonframe}{Magnetismo elettronico}
La maggior parte delle molecole hanno stato fondamentale elettronico $S=L=0$: no momento magnetico. Eccezioni: $\cel{O}{2}{}{}$ con stato fondamentale $S=1$, composti con metalli di transizione, molecole con elettroni dispari.
La maggior parte dei composti sono diamagnetici ($\chi<0$: weak diamagnetis induced by electron orbital current).
Sostanza con momento magnetico in stato fondamentale sono solitamente paramagnetiche ($\chi>0$): se electron magnetic moment interagiscono debolmente \'e possibile EPR-ESR.
(NMR in diamagnetic material: electron magnetism can be ignored because is constant, correzione a campo staticoi)
\end{wordonframe}

\begin{wordonframe}{Kubo: fluctuatuion-dissipation theorem}

\end{wordonframe}

\begin{wordonframe}{Real spectrum FFT: 34-49}
Lorentzian shape-inhomogeneous field broadening-chemical shift(diamagnetic)/spin-spin coupling-J coupling
\end{wordonframe}

\begin{wordonframe}{processi rilassamento spin-spin, spin-lattice}
Lewitt (S dyn): 171-190 nuclear interaction with electric magnetic field (Spin hamiltonian), 543-624 types of relaxation: mechanism, random field relaxation, spectral density, transition probability (dipole-dipole relaxaion: solomon equation, neclear overhauser effectà-)
\end{wordonframe}

\begin{wordonframe}{Nuclear spin interactions}
Spin hamiltonian hypothesis: fast electron motion, low nuclear spin energy.
Nuclear shape, rotation: change in electric/magnetic energies due to orientatio changes.
\end{wordonframe}

\begin{frame}[allowframebreaks]{Free induction decay}

$\pi/2$ pulse lungo $x'$:
\begin{align*}
s(t)\propto\omega_0\int d^3r \exp{-\frac{t}{T_{2}(r)}}B_{\perp}(\vec{r})M_{\perp}(\vec{r},0)\exp{i[(\Omega-\omega(r))t+\phi_0(\vec{r})+\theta_B(\vec{r})]}\\
\phi(\vec{r},t)=-\omega(r)t+\phi_0(\vec{r})=-\gamma B_z(\vec{r})t+\phi_0(\vec{r})
\end{align*}
\begin{figure}[!ht]\includegraphics[trim={0cm 0cm 0 0},clip, keepaspectratio,width=0.4\textwidth]{FIDrep}\label{fig:FIDrep}\end{figure}
\begin{align*}
\frac{1}{T_2*}=\frac{1}{T_2}+\frac{1}{T_2'}\\
M_+(\vec{r},t)=M_+(\vec{r},0)\exp{-\frac{t}{T_2*}}\\
\phi(\vec{r},t)=-\gamma[B_0+\Delta B(\vec{r})]t
\end{align*}
\end{frame}

\begin{frame}[allowframebreaks]{Spin-ECHO and $T_2$ measure.}

\begin{figure}[!ht]\includegraphics[trim={0cm 0cm 0 0},clip, keepaspectratio,width=0.5\textwidth]{SE}\label{fig:SE}\end{figure}
\begin{figure}[!ht]\includegraphics[trim={0cm 0cm 0 0},clip, keepaspectratio,width=\textwidth]{SET2measure}\label{fig:SET2measure}\end{figure}
\begin{align*}
s(T_E)\propto\omega_0\exp{-\frac{T_E}{T_2}}\int d^3 rB_{\perp}M_{\perp}(\vec{r},0)\\
T_2=\frac{T_E'-T_E}{\ln{[s(T_E)/s(T_E')]}}
\end{align*}
\end{frame}

\begin{frame}[allowframebreaks]{Inversion recovery and $T_1$ measure - Spin-echo inversion recovery.}
%succo
Impulso $\pi$ attorno $x'$ e a $T_I$ di $\pi/2$
\begin{align*}
&M_z(0+)=-M_0\\
&M_z(t)=-M_0\exp{-\frac{t}{T_1}}+M_0(1-\exp{-\frac{t}{T_1}})\ 0<t<T_I\\
&\magort{}(t)=|M_0(1-2\exp{-\frac{T_I}{T_1}})|\exp{-\frac{(t-T_I)}{T_2*}}\\
&T_I^{null}=T_1\ln{2}
\end{align*}

\begin{figure}[!ht]\includegraphics[trim={0cm 0 0 0},clip, width=0.9\textwidth]{IR-SE-sampling}\label{fig:IR-SE-sampling}
\end{figure}

\begin{figure}[!ht]\includegraphics[trim={0cm 0 0 0},clip,width=0.9\textwidth]{IRvsTI}\label{fig:IRvsTI}\end{figure}

\end{frame}

\begin{frame}{Chemical shift and NMR spectroscopy}

Broadband RF excitation: complicated FID signal
\begin{align*}
&\omega_{0i}=\gamma_iB_0\\
&\vec{B}_j^{ind}=\delta_j\vec{B}_0
\end{align*}

Shielding constant: linear response of electrons to $B_{ext}$:
\begin{align*}
&B_{shift}(j)=(1-\sigma_j)B_0\to f_{\sigma}=-\sigma\gammabar B_0\\
&s(t)=\sum_jN_j\exp{i\gamma\sigma_jB_0t}
\end{align*}
\end{frame}


\begin{frame}{Fourier imaging}
Aggiungo campo che varia linearmente lungo z:

\begin{align*}
&s(t)\propto\int d^3 r\rho(\vec{r})\exp{i[\Omega t+\phi(\vec{r},t)]}\\
&B_z(z,t)=B_0+zG(t)\ \Rightarrow\ \omega=\omega_0+\omega_G(z,t)
\end{align*}

Frequency encoding: $\omega_G=\gamma zG(t)$.
\begin{equation*}
\phi_G(z,t)=-\int_0^td t'\omega_G(z,t')=-\gamma z\int_0^td t'G(t)
\end{equation*}

1D imaging equation (after demodulation):
\begin{align*}
&s(t)=\int d z\rho(z)\exp{i\phi_G(z,t)}\to s(k)=\int d\,z\rho(z)\exp{-i2\pi kz}\\
&\rho(z)=\int d\,ks(k)\exp{i2\pi kz}
\end{align*}
$\rho$ effective spin density.

\end{frame}

\begin{frame}{Gradient echo}
\begin{figure}[!ht]\includegraphics[trim={0cm 0cm 0 0},clip, keepaspectratio,height=0.9\textheight]{GEimaging}\end{figure}
\end{frame}

\begin{frame}[allowframebreaks]{Spin-echo imaging}
\begin{figure}[!ht]\includegraphics[trim={0cm 0cm 0 0},clip, keepaspectratio,height=0.45\textheight]{SEimaging}\end{figure}
\begin{figure}[!ht]\includegraphics[trim={0cm 0cm 0 0},clip, keepaspectratio,height=0.45\textheight]{SEtotimaging}\end{figure}
\end{frame}

\begin{frame}[allowframebreaks]{Imaging in 3D: 3D imaging  vs 2D multislice imaging}
\begin{columns}[T]
\begin{column}{0.5\textwidth}
3D imaging:
\begin{figure}[!ht]\includegraphics[trim={5cm 0cm 5cm 0},clip, keepaspectratio,width=0.99\textwidth]{3Dimaging}\label{fig:3Dimaging}\end{figure}
\end{column}
\begin{column}{0.49\textwidth}
2D multi-slice imaging:
\begin{figure}[!ht]\includegraphics[trim={0cm 0cm 0 0},clip, keepaspectratio,width=0.99\textwidth]{sliceselection}\label{fig:sliceselection}\end{figure}
\end{column}
\end{columns}
\end{frame}

\begin{frame}[allowframebreaks]{Nyquist criterion, resolution and contrast}
\begin{align*}
&(FOV)^{-1}=\Delta k<\frac{1}{A}\\
&(\Delta k_R=\gammabar\int_t^{t+\Delta t}d t'G_R(t')=\frac{1}{L_R}<\frac{1}{A_R})\\
&f_R=BW_{read}=\frac{1}{\Delta t}=\gammabar G_RL_R>\gammabar G_RA_R
\end{align*}
Risoluzione: $\Delta x=\frac{1}{2n\Delta k}$, $k=(-n\Delta k,(n-1)\Delta k)$.

\begin{align*}
&\SNR/Vxl\propto\frac{\sqrt{N_{aqr}}}{\sigma_0}\quad \Delta t= \frac{1}{\BW_{read}},\ T_s=N_x\Delta t\\
&\SNR/Vxl\propto \frac{\Delta x\Delta y\Delta z\sqrt{N_a}}{\BW_{read}/(N_xN_yN_z)}
\end{align*}
Degrading resolution to improve SNR.

\begin{align*}
&C_{AB}=S_A-S_B\\
&=\rho_{0A}(1-\exp{-\frac{T_R}{T_{1A}}})\exp{-\frac{T_E}{T*_{2A}}}-\rho_{0B}(1-\exp{-\frac{T_R}{T_{1B}}})\exp{-\frac{T_E}{T*_{2B}}}
\end{align*}
Weighting
\end{frame}

\part{Struttura materia tramite NMR}\linkdest{mainconcepts}
\section{Momenti magnetici elementari e magnetizzazione della materia: interazione con campo statico.}\linkdest{magnetizzation}

\begin{frame}{Signal from magnetized material}
%succo
\begin{align*}
&emf=-\TDof{t}\int d^3r\vec{M}(\vec{r},t)\cdot\vec{\mathcal{B}}_{rec}\\
&S\propto \frac{\gamma^3B_0^2\rho_0}{T}
\end{align*}
\end{frame}

\begin{wordonframe}{Magnetizzazione longitudinale all'equilibrio}
La magnetizzazione longitudinale di equilibrio \'e determinata dalla probabilit\'a di occupazione livello energetico $\exp{\exp{-\frac{\scap{m}{B}_0}{KT}}}\approx1-\frac{\scap{m}{B}_0}{KT}$: l'''eccesso'' di spin con con campo $B_0$ \'e $N\frac{\hbar\omega_0}{2KT}$. $\mu=\gamma\vec{J}$, $\gamma=\SI{1.67e8}{\rad\per\tesla\per\second}$ ([\si{\newton\meter\per\tesla}], [\si{\ampere\square\meter}]).
La magnetizzazione risultante \'e $M_0=\rho_0\frac{\gamma^2\hbar^2}{4KT}B_0$. (momento dipolo magnetico per unit\'a di volume)
\end{wordonframe}

\begin{frame}[allowframebreaks]{Equazione del moto dipolo magnetico}
\begin{columns}[T]\begin{column}{0.5\textheight}
\begin{block}{Equazione del moto (bloch0)}
\begin{align*}
&\TDy{t}{\vec{\mu}}=\gamma\vec{\mu}\wedge\vec{B}\\
&|d \mu|=\mu\sin{\theta}d \phi=\gamma\mu B\sin{\theta} dt\\
&\TDof{t}\vec{\mu}^2=0,\ \TDof{t}(\scap{\mu}{B}_0)
\end{align*}
\end{block}
%%succo
\begin{block}{Soluzioni campo B costante: precessione }
\begin{align*}
&\omega=|\TDy{t}{\phi}|=\gamma B\\
&\Rightarrow\ \phi=-\omega_0+\phi_0\\
&\mu_+(t)=\mu_x(t)+i\mu_y(t)\\
&=|\mu_+(t)|\exp{i\phi(t)}
\end{align*}
\end{block}

\end{column}\begin{column}{0.5\textheight}
\begin{block}{Momento magnetico nucleare}
\begin{columns}  \begin{column}{0.05\textwidth}\Pproton\end{column} \begin{column}{0.95\textwidth}
Rapporto giromagnetico: $\vec{\mu}=\gamma\vec{J}$, $\gamma=\SI{2.675e8}{\rad\per\second\per\tesla}$, $\gammabar=\SI{42.58}{\mega\hertz\per\tesla}$.
\end{column}  \end{columns}
\begin{align*}
&\mu_B=\frac{e\hbar}{2m_e}=\SI{9.27e-24}{\ampere\per\square\meter}\\
&\mu_B=\frac{e\hbar}{2m_n}=\SI{5.05e-27}{\ampere\per\square\meter}
\end{align*}
\end{block}
\begin{block}{Momento delle forze}
\begin{align*}
&(\vec{F}=\TDy{t}{\vec{p}})\ d \vec{F}=\vecp{J}{B}\\
&(=Id \vec{l}\wedge\vec{B})\\
&d\vec{N}=\vec{r}\wedge d \vec{F}=\vec{x}\wedge(\vecp{J}{B})
\end{align*}
\end{block}
\end{column}  \end{columns}
\end{frame}

\begin{wordonframe}{spin e dipoli magnetici}

\begin{columns}\begin{column}{0.5\textwidth}
\begin{figure}[!ht]\includegraphics[trim={0cm 0cm 0 0},clip, keepaspectratio,width=0.99\textwidth]{nuclei}\label{fig:nuclei}\end{figure}
\end{column} \begin{column}{0.5\textwidth}
\begin{align*}
&\gamma=\frac{\mu}{J}\\
&\frac{\gamma_n}{2\pi g_n}=\frac{\mu_N}{h}=\SI{7.6}{\mega\hertz\per\tesla}\\
&\gamma_e=-2 \frac{e}{2m_e}\\
&\gamma_p=2.79\frac{e}{2m_p}\\
&\gamma_n=-1.91\frac{e}{2m_n}
\end{align*}
\end{column}\end{columns}
\begin{columns}  \begin{column}{0.5\textwidth}
Atomo con Z elettroni
\begin{align*}
&\vec{\mu}=-g_J \frac{e}{2m}\vec{J}\\
&g_J=\frac{3}{2}+\frac{S(S+1)-L(L+1)}{J(J+1)}
\end{align*}

\end{column} \begin{column}{0.5\textwidth}
Frequenze di Larmor: moto orbitale elettroni $\nu_L=\SI{1.4e10}{\hertz\per\tesla}$, spin elettronico $\nu_L=\SI{2.8}{\mega\hertz\per\gauss}$, spin protonico $\nu_L=\SI{4.3}{\kilo\hertz\per\gauss}$
\end{column}  \end{columns}
\end{wordonframe}

\section{Condizione di risonanza: campo trasverso $B_1$}\linkdest{resonance}

\begin{frame}[allowframebreaks]{Campo a radiofrequenza rotante}
\begin{block}{SR rotante}
\begin{align*}
&\TDy{t}{\vec{\mu}}=(\TDy{t}{\vec{\mu}})'+\vecp{\Omega}{\mu}=\gamma\vecp{\mu}{B}\\
&(\TDy{t}{\vec{\mu}})'=\gamma\vec{\mu}\wedge\vec{B}_{eff}\quad B_{eff}=\vec{B}+\frac{\vec{\Omega}}{\gamma}
\end{align*}
\end{block}
\begin{block}{EOM per campo $B_1$ a RF circolare}
\begin{equation*}
\vec{B}_1^{circ}=B_1(\hat{x}\cos{(\omega t)}-\hat{y}\sin{(\omega t)})
\end{equation*}
fermo nel riferimento rotante: $\vec{\Omega}=-\omega\hat{z}$.
\begin{align*}
&(\TDy{t}{\vec{\mu}})'=\vec{\mu}\wedge[\hat{z}'(\omega_0-\omega)+\hat{x}'\omega_1]=\gamma\vec{\mu}\wedge\vec{B}_{eff}\\
&\omega_0=\gamma B_0,\ \omega=\text{RF lab freq.},\ \omega_1=\gamma B_1
\end{align*}
\end{block}
\begin{block}{on-resonance condition}
\begin{columns}[T]
    \begin{column}{0.5\textheight}
 %%succo
    Per $\omega=\omega_0$:
\begin{align*}
&(\TDy{t}{\vec{\mu}})'=\omega_1\vec{\mu}\wedge\hat{x}'
\end{align*}
Flip-angle: $\Delta\theta=\gamma B_1\tau$.
    \end{column}
    \begin{column}{0.5\textheight}
    RF on-resonance solution
    \begin{align*}
    &\vec{\mu}(t)=R_{x'}(\phi_1(t))\vec{\mu}(0)\\
    &\phi_1(t)=\omega_1t\to\int_{t_0}^td t'\omega_1(t')
    \end{align*}
    \end{column}
\end{columns}
\end{block}
\end{frame}

\begin{wordonframe}{SR rotanti e on-resonance condition (chap 3)}
\begin{equation*}
\TDy{t}{\vec{V}}=(\TDy{t}{V})'+\vecp{\Omega}{V}
\end{equation*}
\begin{block}{Problema 3.3}

\end{block}
\end{wordonframe}

\section{Evoluzione componenti magnetizzazione longitudinale e trasversa: precessione + spin-lattice + spin-spin.}\linkdest{bloch}


\begin{frame}[allowframebreaks]{Magnetizzazione in campo esterno: $T_1$, $T_2$. Equazione di Bloch.}
\begin{columns}[T]
\begin{column}{0.45\textheight}
\begin{block}{Magnetizzazione}
($M=\frac{1}{2}\vecp{x}{J}$)
Momento di dipolo magnetico per unit\'a di volume:
\begin{equation*}
\vec{M}=\frac{1}{V}\sum\vec{\mu}_i
\end{equation*}
V=voxel: volume dove campo magnetico omogeneo, spin hanno stessa fase.
\end{block}
\begin{block}{Dephasing due to $B_{ext}$ inhomogeneities. Spin ECHO}
%succo
$\frac{1}{T_2*}=\frac{1}{T_2}+\frac{1}{T_2}$: se domina $T_2'$ dovuto a disomogeneit\'a compo magnetico esterno $\magort{}$ pu\'o essere rifasata.
\end{block}
\end{column}
\begin{column}{0.55\textheight}
\begin{block}{Interazioni spin-lattice: $T_1$.}
%succo
\begin{align*}
&\TDy{t}{M_z}=\frac{(M_0-M_z)}{T_1}\\
&M_z(t)=M_z(0)\exp{-\frac{t}{T_1}}+M_0(1-\exp{-\frac{t}{T_1}})
\end{align*}
\end{block}
\begin{block}{Local field variation: dephasing. $T_2$}
\begin{align*}
&\TDy{t}{\magort}=(\gamma\magort{}\wedge\vec{B}_{ext})_{NR}-\frac{1}{T_2}\magort{}\\
&\magort{}(t)=\magort{}(0)\exp{-\frac{t}{T_2}}
\end{align*}
\end{block}

\begin{block}{Short/long lived pulses.}
$\Delta\omega\tau_{RF}\geq\frac{1}{4\pi}$. Short: $\Delta\omega_1\gg(\frac{1}{T_1},\frac{1}{T_2})$, ignore decay during pulse. Long: steady state.
\end{block}
\end{column}
\end{columns}
\clearpage Lorentzian shape Mz as fuction of $\omega$: lezione 1 esr
\end{frame}

\begin{wordonframe}{Rilassamento spin-lattice, spin-spin e inomogeneit\'a del campo (chap 4)}
\begin{block}{Densit\'a di energia potenziale magnetica}
Gli spin tendono ad allinearsi col campo magnetico per minimizzare la densit\'a d'energia potenziale, $U_M=-\scap{M}{B}=M_zB_0$: si hanno interazioni spin lattice.
\end{block}
Legge di Curie: $M_0=C\frac{B_0}{T}$,valore di equilibrio.
\begin{block}{Evoluzione $M_z$, inizio arbitrario}
\begin{align*}
&M_z(t)=M_z(0)\exp{-\frac{t}{T_1}}+M_0(1-\exp{-\frac{t}{T_1}})\\
&0\to t_0,\ t\to t-t_0
\end{align*}
\end{block}
\end{wordonframe}

\begin{frame}{Equazione di Bloch}
%succo
    \begin{equation*}
        \TDy{t}{\vec{M}}=\gamma\vec{M}\wedge\vec{B}_{ext}+\frac{1}{T_1}(M_0-M_z)\hat{z}-\frac{1}{T_2}\magort{}
    \end{equation*}
    soluzioni:
    \begin{align*}
&M_x(t)=\exp{-\frac{t}{T_2}}(M_x(0)\sin{(\omega_0t)}+M_y(0)\cos{(\omega_0t)})\\
&M_y(t)=\exp{-\frac{t}{T_2}}(M_y(0)\cos{(\omega_0t)}-M_x(0)\sin{(\omega_0 t)})\\
&M_z(t)=M_z(0)\exp{-\frac{t}{T_1}}+M_0(1-\exp{-\frac{t}{T_1}})\\
&M_+(t)=\exp{-i\omega_0t-t/T_2}M_+(0)\to|M_+(t)|\exp{i\phi(t)}
    \end{align*}
    Per $\vec{B}_{ext}=B_0\hat{z}+B_1\hat{x}$, le equazioni di Bloch sono, con $\Delta\omega=\omega_0-\omega$:
    \begin{align*}
&(\dot{M}_{z'})'=-\omega_1M{y'}+\frac{M_0-M_z}{T_1}\\
&(\dot{M}_{x'})'=\Delta\omega M_{y'}-\frac{M_{x'}}{T_2}\\
&(\dot{M}_{y'})'=-\Delta\omega M_{x'}+\omega_1M_z-\frac{\omega M_{y'}}{T_2}
    \end{align*}
\end{frame}

\begin{frame}{Meccanismi di rilassamento $T_1$}
\begin{columns}  \begin{column}{0.5\textwidth}
dipolo-dipolo: nucleo-nucleo, nucleo-specie paramagnetica.
interazione scalare
\end{column} \begin{column}{0.5\textwidth}

\end{column}  \end{columns}
\end{frame}


\section{NMR in MQ}\linkdest{NMR-MQ}

\begin{wordonframe}{MQ: operatori, autovalori equazione di Schroedinger}
\begin{columns}[T]\begin{column}{0.5\textheight}
\begin{block}{Operatori MQ. Equazione di \scr{}.}
\begin{align*}
&P=-i\hbar\nabla,\ H=-\frac{\hbar^2}{2m}\nabla^2+U\\
&\vec{J}=\vec{L}+\vec{S}\\
&j=|l-s|,\ldots,l+s\\
&j^2=j(j+1)\hbar^2,\ j_z=m_j\hbar\\
&F_z=\mu_zG_z=\mu_z\PDy{z}{B_z}\\
&H\psi=i\hbar\PDy{t}{\psi}:\ \psi=\psi(\vec{r})\exp{-\frac{iEt}{\hbar}}\\
&[p_x,x]=-i\hbar,\ [J_x,J_y]=i\hbar J_z
\end{align*}
\end{block}
\end{column} \begin{column}{0.5\textheight}
\begin{block}{Protone. Spin.}\end{block}
\begin{align*}
&\vec{S}=\frac{1}{2}\hbar\vec{\sigma}\\
&\gamma=g\mu_M,\ \mu_M=\frac{e}{2M}
\end{align*}%succo
Zeeman energy levels: $\Delta E=\frac{1}{2}\gamma B_0\hbar-(-\frac{1}{2}\gamma\hbar B_0)=\hbar\omega_0$.

\end{column}\end{columns}
\begin{equation*}\sigma_i=,\ \sigma_x=\begin{pmatrix}0&1\\1&0\end{pmatrix},\ \sigma_y=\begin{pmatrix}0&-i\\i&0\end{pmatrix},\ \sigma_z=\begin{pmatrix}1&0\\0&-1\end{pmatrix}
 \end{equation*}
\end{wordonframe}

\begin{frame}{Soluzione equatione di Schroedinger per spin in campo magnetico costante}
\begin{align*}
&H\psi=E\phi,\ H=-\scap{\mu}{B}\text{Per j fissato:}\\
&\psi(\vec{r},t)=\sum_{m_j=-j}^jC_{m_j}\psi_{}(\vec{r})\exp{-\frac{i}{\hbar}E_{m_j}t}\ E_{m_j}=-\gamma m_j\hbar B_0
\end{align*}
\begin{block}{Precessione attorno a $B_0$}

\begin{columns}[T] \begin{column}{0.6\textheight}
Per protone: $H=-\gamma B_0S_z=\begin{pmatrix}-1/2\hbar\omega_0&0\\0&1/2\hbar\omega_0\end{pmatrix}$.
\begin{align*}
&\braket{\psi/\vec{\mu}|\psi}=\int\psi*\vec{\mu}\psi d V=\psi*\vec{\mu}\psi V\\
&\exv{\mu_x}=\frac{\gamma\hbar}{2}\sin{\theta}\cos{(\phi_0-\omega_0t)}\\
&\exv{\mu_y}=\frac{\gamma\hbar}{2}\sin{\theta}\sin{(\phi_0-\omega_0t)}\\
&\exv{\mu_z}=\frac{\gamma\hbar}{2}\cos{\theta}
\end{align*}
\end{column}
\begin{column}{0.4\textheight}
Soluzione: $\psi_+=\begin{pmatrix}1\\0\end{pmatrix}$/$\psi_-=\begin{pmatrix}0\\1\end{pmatrix}$, $E_{\pm}=\mp \frac{1}{2}\hbar\omega_0$.
Il valore di aspettazione della magnetizzazione \'e un vettore di ampiezza $\frac{\gamma\hbar}{2}$ che precede con velocit\'a $\omega_0$.
\end{column}\end{columns}
\end{block}

\end{frame}

\begin{wordonframe}{Valore di aspettazione momento magnetico protone in campo esterno}
\begin{align*}
&\braket{\psi|\vec{\mu}|\psi}=\gamma\sum_{m,m'}C_mC_{m'}\dag\psi_{m'}\dag\vec{S}\psi_m \exp{-\frac{i(E_{m'}-E_m)t}{\hbar}}\\
&\psi_{m'}\dag|vec{\sigma}\psi_m=\hat{x}\delta_{m',m}+2mi\hat{y}\delta_{m',-m}+2m\hat{z}\delta_{m',m}\\
&V\sum|C_m|^2=1,\ C_{\pm}=a_{\pm}\exp{i\alpha_{\pm}}\\
&a_+=\frac{\cos{\Theta}}{\sqrt{V}},\ a_-=\frac{\sin{\Theta}}{\sqrt{V}},\ \Theta=\frac{\theta}{2},\ \phi_0=\alpha_--\alpha_+
\end{align*}

\end{wordonframe}

\begin{frame}{RF spin tipping (ribaltamento)}
\begin{block}{RF field}
\begin{equation*}
H(t)=-\scap{\mu}{B}(t)=\frac{\gamma\hbar}{2}(\sigma_zB_0+(\sigma_x\cos{\omega t}-\sigma_y\sin{\omega t})B_1)
\end{equation*}
\end{block}
\begin{block}{Soluzione equazione \scr{} in riferimento rotante}
\begin{equation*}\Psi=\psi_1'(t)\psi_+\exp{i\omega_0t/2}+\psi_2'(t)\psi_-\exp{-i\omega_0t/2}\end{equation*}
\end{block}
\begin{align*}
&\exv{\mu_x(t)}=\frac{\gamma\hbar}{2}\sin{\Theta}\cos{\Phi}\\
&\exv{\mu_y(t)}=\frac{\gamma\hbar}{2}[\cos{\Theta}\sin{(\omega_1t)}+\sin{\Theta}\cos{(\omega_1t)}\sin{\Phi}]\\
&\exv{\mu_z(t)}=\frac{\gamma\hbar}{2}[\cos{\Theta}\sin{\omega_1t}-\sin{\theta}\sin{(\omega_1t)}]\\
&\Phi=\phi_1-\phi_2-\pi/2
\end{align*}

\end{frame}

\begin{wordonframe}{Spin tipping in rotating frame}
In riferimento rotante, la componente di $\vec{\sigma}$ su $\hat{x'}$ e $\hat{y'}$:
\begin{align*}
&\sigma_{x'}=\sigma_{x}\cos{\omega t}-\sigma_{y}\sin{\omega t}=\begin{pmatrix}0&\exp{i\omega t}\\\exp{-i\omega t}&0\end{pmatrix}\\
&\sigma_{y'}=\sigma_{x}\sin{\omega t}+\sigma_{y}\cos{\omega t}=\begin{pmatrix}0&-\exp{i\omega t}\\\exp{-i\omega t}&0\end{pmatrix}
\end{align*}
Soluzioni nel SR rotante:
\begin{align*}
&\psi_1'=c_1\cos{\frac{\omega_1t}{2}}+c_2\sin{\frac{\omega_1t}{2}}\\
&\psi_1'=c_3\cos{\frac{\omega_1t}{2}}+c_4\sin{\frac{\omega_1t}{2}}\\
&c_3=-ic_2,\ c_4=ic_1\\
&c_1=\frac{1}{\sqrt{V}}\cos{\Theta/2}\exp{-i\phi_1}\\
&c_2=\frac{1}{\sqrt{V}}\sin{\Theta/2}\exp{-i\phi_2}
\end{align*}
\end{wordonframe}

\begin{frame}{Equilibrium properties: spin excess and magnetization}
Per $\frac{\hbar\omega}{KT}\ll1$ si ha spin excess $N\frac{\hbar\omega_0}{2KT}$. 
$M_0=\frac{\rho_0\gamma^2\hbar^2}{4KT}B_0$
\end{frame}


\section{Impulsi e segnali}\linkdest{impulse}

\begin{frame}{Signal detection}
\begin{align*}
&emf=-\TDy{t}{\Phi_{coil}}=\TDof{t}\int\,d^3r\vec{M}(\vec{r},t)\cdot\vec{\mathcal{B}}_{rec}\\
&S\propto \frac{\gamma^3B_0^2\rho_0}{T}\\
&\mathcal{B}^{rec}=\frac{\vec{B}(\vec{r}')}{I}
\end{align*}
\end{frame}

\begin{frame}[allowframebreaks]{FID, Spin echo and $T_2$ measure}
\begin{block}{Free induction decay}%
$\pi/2$ pulse lungo $x'$:
\begin{align*}
&s(t)\propto\omega_0\int d^3r \exp{-\frac{t}{T_2(r)}}B_{\perp}(\vec{r})M_{\perp}(\vec{r},0)\exp{i[(\Omega-\omega(r))t+\phi_0(\vec{r})+\theta_B(\vec{r})]}\\
&\phi(\vec{r},t)=-\omega(r)t+\phi_0(\vec{r})=-\gamma B_z(\vec{r})t+\phi_0(\vec{r})
\end{align*}
Possibilit\'a che lòa frequenza di precessione cambi con la posizione $\omega(\vec{r})$
%B_x=B_{\perp}\cos{\theta_b}
\end{block}
\begin{block}{Repeated FID}
\begin{figure}[!ht]\includegraphics[trim={0cm 0cm 0 0},clip, keepaspectratio,width=0.4\textwidth]{FIDrep}\label{fig:FIDrep}\end{figure}
\end{block}
\end{frame}

\begin{frame}{$T_2*$ decay: dephasing of magnetization.}
%succo
\begin{align*}
&\frac{1}{T_2*}=\frac{1}{T_2}+\frac{1}{T_2'}\\
&M_+(\vec{r},t)=M_+(\vec{r},0)\exp{-\frac{t}{T_2*}}\\
&\phi(\vec{r},t)=-\gamma[B_0+\Delta B(\vec{r})]t
\end{align*}
$T_2'$ is machine/sample dependent $B_0$ inhomogeneities and is recoverable, $T_2$ is thermodynamic relaxing: $\sum_{sample}\exp{i\phi(\vec{r},t)}\to0$. Often $T_2'\ll T_2$.

\begin{block}{$T_2$ measure. Spin-echo.}
Impulso $(\pi/2)_{x'}$ determina spin (excess) lungo $y'$. $\Delta B(\vec{r})\neq0$ implica dephasing:
\begin{equation*}
\phi(\vec{r},t)=-\gamma\Delta B(\vec{r})t\quad 0\leq t\leq\tau
\end{equation*}
Applico impulso di $\pi$ attorno a $y'$:
\begin{align*}
&\phi(\vec{r},\tau+)=-\phi(\vec{r},\tau-)=\gamma\Delta B(\vec{r})\tau\\
&\phi(\vec{r},t)=-\phi(\vec{r},\tau)-\gamma\Delta B(\vec{r})(t-\tau)=-\gamma\Delta B(\vec{r})(t-t_E)
\end{align*}
\end{block}
\begin{block}{Misura $T_2$.}
%succo
\begin{align*}
&s(T_E)\propto\omega_0\exp{-\frac{T_E}{T_2}}\int d^3 rB_{\perp}M_{\perp}(\vec{r},0)\\
&T_2=\frac{T_E'-T_E}{\ln{[s(T_E)/s(T_E')]}}
\end{align*}
\end{block}
\end{frame}

\begin{wordonframe}{Spin-ECHO}
\begin{block}{Spin-echo exponential}
\begin{align*}
&(\TDy{t}{(M_+)})'=-R_2^{se}M_+\\
&R_2^{se}=\begin{pmatrix}R_2'+R_2\\-R_2'+R_2\\R_2'+R_2\end{pmatrix}\\
&M_{\perp}(t)=M_{\perp}(t_0)\exp{-(t-t_0)R_2^{se}}\\
&=M_{\perp}(0)\begin{pmatrix}\exp{-\frac{r}{T_2*}}\ 0<t<\tau\\\exp{-\frac{t}{T_2}}\exp{-\frac{(T_E-t)}{T_2'}}\ \tau<t<2\tau=T_E\\ \exp{-\frac{t}{T_2}}\exp{-\frac{(t-T_E)}{T_2'}}=\exp{-\frac{t}{T_2*}}\exp{-\frac{T_E}{T_2'}} \end{pmatrix}
\end{align*}
$T_2$ intrinsic rapid time fluctuation of local field.
\end{block}
Misure con $T_R\leq T_1$: ottimizzazione.
\end{wordonframe}

\begin{frame}[allowframebreaks]{Inversion recovery and $T_1$ measure - Spin-echo inversion recovery.}
Impulso $\pi$ attorno $x'$ e a $T_I$ di $\pi/2$
\begin{align*}
&M_z(0+)=-M_0\\
&M_z(t)=-M_0\exp{-\frac{t}{T_1}}+M_0(1-\exp{-\frac{t}{T_1}})\ 0<t<T_I\\
&\magort{}(t)=|M_0(1-2\exp{-\frac{T_I}{T_1}})|\exp{-\frac{(t-T_I)}{T_2*}}\\
&T_I^{null}=T_1\ln{2}
\end{align*}

\begin{figure}[!ht]\includegraphics[trim={0cm 0 0 0},clip, width=0.9\textwidth]{IR-SE-sampling}
\label{fig:IR-SE-sampling}\end{figure} 
\begin{figure}[!ht]\includegraphics[trim={0cm 0 0 0},clip,width=0.9\textwidth]{IRvsTI}
\label{fig:IRvsTI}
\end{figure}

\end{frame}

\begin{frame}{Chemical shift and NMR spectroscopy}
%succo
Broadband RF excitation: complicated FID signal
\begin{align*}
&\omega_{0i}=\gamma_iB_0
\end{align*}
Determinazioni specie nucleari nel campione (excited by pulse spectrum).
Shielding constant: linear response of electrons to $B_{ext}$:
\begin{align*}
&B_{shift}(j)=(1-\sigma_j)B_0\to f_{\sigma}=-\sigma\gammabar B_0\\
&s(t)=\sum_jN_j\exp{i\gamma\sigma_jB_0t}
\end{align*}
\end{frame}

\section{Tecniche di rivelazione}\linkdest{tech}

\begin{frame}{Classi di rivelatori}
\begin{itemize}
\item Passivo in onda continua: generatore separato dal circuito LC. Induzione di Bloch, Q-meter, metodi a ponte
\item Attivi in onda continua: il circuito risonante contenente il campione fa parte di un oscillatore la cui ampiezza varia al passaggio dalla risonanza. Marginale, Robinson.
\item Tecniche passive impulsive: inducono nel campione stati di magnetizzazione di non equilibrio per determinare i tempi di rilassamento (ritorno a equilibrio termodinamico).
\end{itemize}
\end{frame}

\subsection{CW NMR}

\begin{frame}{CW measurement}
\begin{columns}  \begin{column}{0.5\textwidth}
\begin{align*}
&I(t)=\frac{V_0\cos{\omega t+\phi}}{Z}\\
&Z=\sqrt{R^2+(X_L-X_C)^2}\ \phi=\arctan{\frac{X_L-X_C}{R}}\\
&V_C=X_CI=\frac{|V_0|}{\sqrt{\tau^2\omega^2+(\frac{\omega^2}{\omega_0^2}-1)^2}}\ \tau=RC
\end{align*}
\end{column} \begin{column}{0.5\textwidth}
Guadagno
\begin{align*}
&|G|^2=\frac{V_C^2}{V_0^2}\\
&=\frac{1}{(\frac{1}{Q})^2 \frac{\omega^2}{\omega_0^2}+(\frac{\omega^2}{\omega_0^2}-1)^2}\\
&=\frac{\omega_0^4}{\omega^2\delta^2+(\omega^2-\omega_0^2)^2}
\end{align*}
$\delta=R/L$ 
\end{column}  \end{columns}
$\chi(\omega)=\chi'+i\chi''$
\end{frame}

\section{Imaging: phase encoding, k-spazio, imaging equation}\linkdest{imaging}

\begin{frame}[allowframebreaks]{Fourier imaging, gradient echo and K-space}
Rilassamento dopo $\pi/2$-pulse:$M_{\perp}=M_0(\vec{r})$.
\begin{block}{Aggiungo campo che varia linearmente lungo z}
%succo
\begin{align*}
&s(t)\propto\int d^3 r\rho(\vec{r})\exp{i[\Omega t+\phi(\vec{r},t)]}\\
&B_z(z,t)=B_0+zG(t)\ \Rightarrow\ \omega=\omega_0+\omega_G(z,t)
\end{align*}
Frequency encoding: $\omega_G=\gamma zG(t)$.
\begin{equation*}
\phi_G(z,t)=-\int_0^td t'\omega_G(z,t')=-\gamma z\int_0^td t'G(t)
\end{equation*}
\end{block}
\begin{block}{1D imaging equation}
%succo
\begin{equation*}
s(t)=\int d z\rho(z)\exp{i\phi_G(z,t)}
\end{equation*}
$\rho$ effective spin density
\end{block}
\begin{figure}[!ht]\includegraphics[trim={0cm 0cm 0 0},clip, keepaspectratio,width=\textwidth]{FID-k}\label{fig:FID-k}\end{figure}
\end{frame}

\begin{wordonframe}{Effective spin density}
\begin{align*}
M_0=\frac{1}{4}\rho_0(\vec{r})\frac{\gamma^2\hbar^2}{KT}\\
\rho(\vec{r})=\omega_0\Lambda B_{\perp}M_0
\end{align*}
\end{wordonframe} 

\begin{frame}{K-spazio}
\begin{align*}
s(k)=\int d z\rho(z)\exp{-i2\pi kz}
\end{align*}
Per gradiente uniforma lungo z $s(k)$ \'e la trasformata di Fourier della "densit\'a effettiva di spin del campione.
\begin{block}{Antitrasformata del segnale \'e la densit\'a spaziale}
\begin{equation*}
\rho(z)=\int d ks(k)\exp{i2\pi kz}
\end{equation*}
\end{block}
\begin{block}{Coverage of K-space}
Campionamento uniforme: sampling at constant rate in presence of constant gradient.
\begin{equation*}
k=\gammabar\int_0^tG(t')d t'
\end{equation*}
\end{block} 
\end{frame}

\begin{frame}{two spin system}
Two spin at $z=\pm z_0$: in rotating frame they preceed clock/anticlock-wise.
\begin{align*}
&\phi(\pm z_0,t)=\mp\gamma Gz_0(t-t_1)\\
&s(t)=s_0[\exp{-i\gamma Gz_0t}+\exp{i\gamma Gz_0t}]\\
&=2s_0\cos{(\gamma G t z_0)}=2s_0\cos{(2\pi kz_0)}\\
&0\leq t\leq t_2:\ 0\leq k\leq k_2=\gammabar Gt_2
\rho(z)=\intsinf{}d k2s_0\cos{(2pi kz_0)}\exp{i2\pi kz}\\
&=s_0[\delta(z+z_0)+\delta(z-z_0)]
\end{align*}

\end{frame}

\begin{frame}{Gradient echo}
%succo
\begin{figure}[!ht]\includegraphics[trim={0cm 0cm 0 0},clip, keepaspectratio,height=0.9\textheight]{GEimaging}\end{figure}

\end{frame}

\begin{wordonframe}{Gradient echo}
\begin{align*}
&s(t')=\int d z\rho(z)\exp{-i\gamma Gzt'}:\ -\frac{t_4-t_3}{2}\leq t'\leq \frac{t_4-t_3}{2}
\end{align*}
\end{wordonframe}

\begin{frame}[allowframebreaks]{Spin-echo imaging}
%succo
\begin{figure}[!ht]\includegraphics[trim={0cm 0cm 0 0},clip, keepaspectratio,height=0.45\textheight]{SEimaging}\end{figure}
\begin{figure}[!ht]\includegraphics[trim={0cm 0cm 0 0},clip, keepaspectratio,height=0.45\textheight]{SEtotimaging}\end{figure}
\end{frame}

\begin{wordonframe}{Spin echo imaging}
Gradient echo refocus phase induced by gradient but does NOT refocus dephasing due static field inhomogeneities.

\begin{align*}
&\phi(z,t)=-\gamma\Delta B(z)t-\gamma Gz(t-t_1):\ t_1<t<t_2\\
&(\pi)_{y'},\ t=\tau:\ \phi\to-\phi\\
&\phi(z,t)=\gamma\Delta B(z)\tau+\gamma Gz(t_2-t_1)-\gamma\Delta B(z)(t-\tau)\\
&-\gamma Gz(t-t_3):\ t_3<t<t_4
\end{align*}
Total phase induced by $B_0/G$ inhomogeneities has been refocused.
Si ha:
\begin{align*}
&\phi(z,T_E)=0:\ 2\tau=t_3+(t_2-t_1)
\end{align*}
\end{wordonframe}


\begin{frame}{K-space coverage}
\begin{figure}[!ht]\includegraphics[trim={0cm 5cm 0 0},clip, keepaspectratio,height=0.3\textheight]{FID-k}\end{figure}

\begin{figure}[!ht]\includegraphics[trim={0cm 0cm 0 0},clip, keepaspectratio,height=0.6\textheight]{SE-GE-k}\end{figure}
\end{frame}

\begin{frame}[allowframebreaks]{Imaging in 3D: 3D imaging  vs 2D multislice imaging}
\begin{block}{3D imaging}
\begin{figure}[!ht]\includegraphics[trim={0cm 0cm 0 0},clip, keepaspectratio,height=0.5\textheight]{3Dimaging}\label{fig:3Dimaging}\end{figure}
\begin{align*}
&s(\vec{k})=\int d^3 r\rho(\vec{r})\exp{-i2\pi \scap{k}{r}}\\
&k_i(t)=\gammabar\int_0^tG_i(t')d t'\\
&\mathcal{F}s(\vec{k})\propto\rho(\vec{r})
\end{align*}
\end{block}
\begin{block}{2D multi-slice imaging}
Slice selection, $f(z)=f_0+\gammabar G_zz$: $z_0\pm\frac{\Delta z}{2}$ quindi $f_{RF}=\gammabar G_zz_0-\gammabar G_z\frac{\Delta z}{2}\gammabar G_zz_0+\gammabar G_z\frac{\Delta z}{2}$, cio\'e $BW_{RF}=\Delta f$. Imponiamo che nella slice il flip-angle sia uniforme
\begin{align*}
&RF(f)=\rect{\frac{\Delta f}{f}}:\ -\gammabar G_z\frac{\Delta z}{2}\leq f\leq \gammabar G_z\frac{\Delta z}{2}\\
&B_1(t)=\sinc{(\pi\Delta f t)}
\end{align*}

\begin{figure}[!ht]\includegraphics[trim={0cm 0cm 0 0},clip, keepaspectratio,width=0.3\textwidth]{sliceGBW}\label{fig:sliceGBW}\end{figure} 

Rephasing gradient: $\frac{|\int d tG_{rephase}|}{|\int d tG_{SS}|}=50\%$.

\begin{figure}[!ht]\includegraphics[trim={0cm 0cm 0 0},clip, keepaspectratio,width=0.4\textwidth]{sliceselection}\label{fig:sliceselection}\end{figure}

\end{block}
\end{frame}

\section{Noise and contrast}\linkdest{noisecontrast}

\begin{frame}{Nyquist criterion and resolution}
$\frac{1}{\Delta k}=L=FOV$, A object size:
\begin{align*}
&\Delta k<\frac{1}{A}\\
&(\Delta k_R=\gammabar\int_t^{t+\Delta t}d t'G_R(t')=\frac{1}{L_R}<\frac{1}{A_R})\\
&f_R=BW_{read}=\frac{1}{\Delta t}=\gammabar G_RL_R>\gammabar G_RA_R
\end{align*}
Risoluzione: $\Delta x=\frac{1}{2n\Delta k}$, $k=(-n\Delta k,(n-1)\Delta k)$.
\end{frame}

\begin{frame}[allowframebreaks]{SNR e contrasto}
\begin{block}{Signal to Noise ratio}
$SNR=\frac{s}{\sigma_{noise}}$:
\begin{align*}
&\sigma^2(emf_{noise})-\sigma_{TH}^2\propto\bar{(emf_{noise}-\bar{emf_{noise}})^2}=4KTR_EBW
\end{align*}
$R_E$ resistenza effettiva coil+body.
\begin{align*}
&\SNR/Vxl\propto\frac{\sqrt{N_{aqr}}}{\sigma_0}\quad \Delta t= \frac{1}{\BW_{read}},\ T_s=N_x\Delta t\\
&\SNR/Vxl\propto \frac{\Delta x\Delta y\Delta z\sqrt{N_a}}{\BW_{read}/(N_xN_yN_z)}
\end{align*}
Degrading resolution to improve SNR.
\end{block}
\begin{block}{Contrasto tessuto A/B}
\begin{align*}
&C_{AB}=S_A-S_B\\
&=\rho_{0A}(1-\exp{-\frac{T_R}{T_{1A}}})\exp{-\frac{T_E}{T*_{2A}}}-\rho_{0B}(1-\exp{-\frac{T_R}{T_{1B}}})\exp{-\frac{T_E}{T*_{2B}}}
\end{align*}
Spin density weighting:
\begin{equation*}
\left\{\begin{pmatrix}T_E\ll T_{2A,B}^*\\ T_R\gg T_{1A,B}\end{pmatrix}\right.\Rightarrow C_{AB}\approx\rho_{0A}-\rho_{0B}
\end{equation*}
$T_1$-weighting: $T_E\ll {T_2^*}_{AB}$.
$T_2^*$-weighting: $T_R\gg {T_1}_{AB}$.
\end{block}
\end{frame}



%\part{Basi NMR: nuclei, spin, campi magnetici e magnetizzazione della materia}\label{part:}
%\input{NMR}

\end{document}