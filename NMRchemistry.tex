\section{Nuclei in campi magnetici}

\begin{frame}{Spin nucleari}
    
\end{frame}


\begin{frame}{Momenti dipolo magnetico}
    $\frac{e\hbar}{2m_N}=\SI{3.1525e-8}{\ev\per\tesla}$, $\hbar=\SI{6.58e-16}{\ev\per\second}$, ($\gamma=\frac{g\mu_N}{\hbar}$)
\end{frame}

\begin{frame}{Precessione momento dipolo in campo magnetico}
   $\vec{\mu}=g\vec{l}\mu_N$, $\nu_{Lar}=\frac{\gamma B_0}{2\pi}$.
   Equazione di moto: $\TDy{t}{\vec{\mu}}=\vecp{\mu}{B}$.
(tab nucleispin $\gamma$: Nmr spectroscopypg10)
\end{frame}

\begin{frame}{Magnetizzazione del campione: longitudinal equilibrium magnetization}
Boltzmann distribution: $\frac{P(m=-1/2)}{P(m=+1/2)}=\exp{-\Delta E/kT}$, $\Delta E=\hbar\gamma B_0$: spin excess is $N\frac{\hbar\omega_0}{2kT}$ then $M_0=\rho_0\frac{\hbar^2\gamma^2}{4kT}B_0$
\end{frame}

\begin{frame}{Coordinate cartesiane}
\begin{columns}[T]
\begin{column}{0.5\textwidth}
\begin{align*}
&|d\vec{\mu}|=\mu\sin{\theta}|d\phi|\\
&|d\vec{\mu}|=\gamma|\vecp{\mu}{B}|\,dt=\gamma\mu B\sin{\theta}\,dt\\
&\Rightarrow\omega=|TDy{t}{\phi}|=\gamma B,\ \phi=-\omega_0t+\phi_0
\end{align*}
\end{column}
\begin{column}{0.5\textwidth}
\begin{align*}
&\mu_x(t)=\mu_x(0)\cos{\omega_0t}+\mu_y(0)\sin{\omega_0t}\\
&\mu_y(t)=\mu_y(0)\cos{\omega_0t}-\mu_x(0)\sin{\omega_0t}\\
&\mu_x(t)=\mu_z(0)\\
&\mu_+(t)=\mu_+(0)\exp{-i\omega_0t}
\end{align*}
\end{column}
\end{columns}

\end{frame}

\begin{frame}{Detecting system magnetization}
RF magnetic field at Larmor frequency rotate magnetization away from alignment with $B_0$:
\end{frame}

\begin{frame}{Precessione magnetizzazione: sistema di riferimento rotante}

\end{frame}

\begin{frame}{Precessione magnetizzazione: campo rotante}
   Rotating magnetic field:
   \begin{equation*}
   \vec{B}(t)=B_0\hat{z}+B_1(\hat{x}\cos{\omega t}-\hat{y}\sin{\omega t})
   \end{equation*}
\end{frame}

\section{NMR spectroscopy}

\begin{frame}{Shape of the spectrum}
    \begin{align*}
&s(t)=\sum_ls_l(t)\\
&s_l(t)=a_l\exp{(i\omega_0(l)-\lambda(l))t}\\
&S_l(\omega)=\int_0^{\infty}s_l(t)\exp{-i\omega t}d\,t\\
&=a_l\frac{1}{\lambda(l)+i(\omega-\omega(l))}
    \end{align*}
    SI definiscono Lorenziana di assorbimento e dispersione
    \begin{align*}
    &A=\frac{\lambda}{\lambda^2+(\omega-\omega(l))^2}\\
    &D=-\frac{\omega-\omega(l)}{\lambda^2+(\omega-\omega(l))^2}
    \end{align*}
\end{frame}

\section{Molecular structure}

\begin{frame}{Chemical shift and other magnetic interactions}
\chemfig{**6(------)} \hspace{.5mm} + \hspace{.5mm} \chemfig{H_3C-Cl} \hspace{.5cm} $\xrightarrow{catalyst}$ \hspace{.5mm} \chemfig{**6(---(-)---)} \hspace{.5mm} + \hspace{.5mm} \chemfig{H-Cl}
\end{frame}

\begin{frame}{Aminoacidi}
Amin($NH_2$)-acido(COOH)-
\end{frame}

\begin{frame}{Benzene: aromatic shielding}
    
\end{frame}

\begin{frame}{Accoppiamento scalare}
    
\end{frame}